% --------------------------------------------------------------
% This is all preamble stuff that you don't have to worry about.
% Head down to where it says "Start here"
% --------------------------------------------------------------
 
\documentclass[letterpaper,12pt]{article}
\usepackage[margin=1in]{geometry} 
\usepackage{amsmath,amsthm,amssymb}
\usepackage[margin=1in]{geometry} 
\usepackage{amsmath,amsthm,amssymb}
\usepackage[T1]{fontenc} %escribe lo del teclado
\usepackage[utf8]{inputenc} %Reconoce algunos símbolos
\usepackage{lmodern} %optimiza algunas fuentes
\usepackage{graphicx}
\graphicspath{ {images/} }
\usepackage{hyperref} % Uso de links
\usepackage[colorinlistoftodos]{todonotes}


\newcommand{\N}{\mathbb{N}}
\newcommand{\Z}{\mathbb{Z}}
 
\newenvironment{theorem}[2][Theorem]{\begin{trivlist}
\item[\hskip \labelsep {\bfseries #1}\hskip \labelsep {\bfseries #2.}]}{\end{trivlist}}
\newenvironment{lemma}[2][Lemma]{\begin{trivlist}
\item[\hskip \labelsep {\bfseries #1}\hskip \labelsep {\bfseries #2.}]}{\end{trivlist}}
\newenvironment{exercise}[2][Exercise]{\begin{trivlist}
\item[\hskip \labelsep {\bfseries #1}\hskip \labelsep {\bfseries #2.}]}{\end{trivlist}}
\newenvironment{problem}[2][Problem]{\begin{trivlist}
\item[\hskip \labelsep {\bfseries #1}\hskip \labelsep {\bfseries #2.}]}{\end{trivlist}}
\newenvironment{question}[2][Question]{\begin{trivlist}
\item[\hskip \labelsep {\bfseries #1}\hskip \labelsep {\bfseries #2.}]}{\end{trivlist}}
\newenvironment{corollary}[2][Corollary]{\begin{trivlist}
\item[\hskip \labelsep {\bfseries #1}\hskip \labelsep {\bfseries #2.}]}{\end{trivlist}}

\newenvironment{solution}{\begin{proof}[Solution]}{\end{proof}}
 
\begin{document}
 
% --------------------------------------------------------------
%                         Start here
% --------------------------------------------------------------
 
\title{READING NOTES: Technological Determinism and Functionalism}
\author{Zhilong Wang}

\maketitle
\section{Summary}
Karl Marx is a pioneer of systematic social theory. 
He thinks that the technology dominates the development of society.
Deeply influenced by Enlightenment, who believed that the future of society 
will became better, Marx thought that people made their own society by the production 
of material life, such as, food, clothes, and shelter\cite{textbook}.
In Marx's opinion, the society is in a dynaimc state rather than a stationary state.
Change of the production methods will change the social struction and the relation between people.

To demonstrate his theory, Marx introduced the concept of \textbf{relation of production}
as the economic base of society. 
Relation of production is the economic base, it means how production is organized and what technologies are utilized.
The economic base determinates the superstructure---society structure and political sturcture.
In different social structures, people has different roles in the relation of production.
Each roles can be clarified into two classes: the ruling class and subject class.
The evolution of society is drived by the conflict between the ruling class and subject class.

Marx devided the history into three phases: handicraft, manufacturing, and modern industry.
Each phrases has a different technologies of production, relations of production and social sturcture.

In the phase of handicraft, a single worker made the entire article\cite{textbook}.
In the phase of manufacturing, wokers produced the production in the factory in a collaboration way.
In the phase of modern industry, goods were produced by the power-driven machine, 
Workers became the appendage to the machine and bourgeois generally dominated the production.

\newcommand{\tabincell}[2]{\begin{tabular}{@{}#1@{}}#2\end{tabular}}  

\noindent
\begin{table}
    \caption{Mode and Relation of Production}
  \centering
\begin{tabular}{|l|*{3}{c|}}\hline
&\tabincell{c}{Production\\ scale}&How to pruduce& Social classes\\\hline\hline
Handicraft & Small & \tabincell{c}{A single worker makes\\ the entire article}& Master person \& Worker\\\hline
Manufacturing &Greater&\tabincell{c}{Collaboration and\\ Assembly line}& The bourgeoisie \& Woker\\\hline
Modern industry &Very big&\tabincell{c}{Power-driven\\ Procuditon} & Employees \& Employers\\\hline
\end{tabular}
\end{table}

\begin{table}
    \caption{Technology that Drive the Restructuring of Society}
  \centering
\begin{tabular}{|l|*{3}{c|}}\hline
From&&To& Domination\\\hline\hline
Handicraft & $\rightarrow$ & Manufacturing & Machine spinning and Machine weaving \\\hline
Manufacturing &$\rightarrow$&Modern industry& Power-driven Machine \& Woker\\\hline
\end{tabular}
\end{table}

Besides, Marx's theory supposed that the production process does not only creates things, 
but also produces the social relation.
"In production, a men not only act on nature but also on one another."\cite{marx1978marx}
Five theoretical concepts in Marx's theory are important to comprehend 
contemporary industry and their impact upon human experience:


\begin{itemize}
  \item Labour-power: the capacity to work. The worker sells their labour-power for wages.
  \item Surplus-value: labour-power can be put to work to create value greater than the wages be paied, that is the surplus-value.
  Capitalists usually called surplus value as ‘profit’.
  \item Use-value: use-value is a qualitative measure that represents the goods value of satisfing the human's needs.
  \item Exchange value: exchange value is quantitative measure that 
  based on the average socially necessary labour-time needed to produce an object at a given level of human competency and technical proficiency\cite{textbook}.
  Expressed simply, we usually call the exchange value as 'price'.
  \item Commodity fetishism. 
  While Mark's theoy regards the exchange of goods as the relation between humans as creators and exchangers, 
  the commodity fetishism sees the exchange of goods as relations between things.
\end{itemize}

Technological determinists is a theory that sees technology as the driving force of history.
Techonlogy exists outside of social relations and drives the reconstructing of social structure.\cite{textbook}.
Marx’s theory provides a materialist approach to understanding of society.
Frankfurt School and the Forces of Production are two extensions of Marx's theory.


Different countries, orders and powers become more and more influencial in a certain bound to the global 
change through new digital media. 
New media technology does not only shapes society, but also increases the power and control from the government 
to social environment.
It changes the politics, culture and consumerism in different way.


Merton's theory of social structure of science proposed four ethos of science in 1942:
\textbf{universalism}, \textbf{communism},\textbf{disinterestedness} and \textbf{organized skepticism}.
In addition to these ethos of science, there are also some ethics of science.

There are two important question concerning technology:
The first question is "Is technology applied science",
the second question is "Does technology drive history".
Regarding the first question, though most people agree with it,
some areas show that scientific knowledge does not play the most important 
role in the development of many state-of-art technologies.
Actually, the the development of technology integrates many different categaries of knowledge, 
there are multiple relations of science and technology, rather than a single monolithic relation.
Regarding the second question, Bijker supposed that 
"purely social relations are to be found only in the imaginations of sociologists or among baboons. 
But equally, technology could be almost nothing without history.
and purely technical relations are to be found only in the wilder reaches of science fiction"\cite{pinch1984social}.




\section{Critique}
I think that the Mark's theory well deminstrates relation and interact 
between the technology, material, and social relaton, even though it has some limitation 
because of the historical circumstance in which Marx lived.
In my opinion, we should regard Mark's theory as a way that how we observe the world 
rather than a dogmas. 
The history shown us that the society, political structure, culture and techonlogy 
could influence each other. We can neither say that it is the technology determinates the society 
nor the development of society drive the development of technology.
There are so complex relation among them that we may need to think them as a whole.

% --------------------------------------------------------------
%     You don't have to mess with anything below this line.
% --------------------------------------------------------------
\bibliographystyle{IEEEtranS}
\bibliography{ref}
\end{document}
% --------------------------------------------------------------
% This is all preamble stuff that you don't have to worry about.
% Head down to where it says "Start here"
% --------------------------------------------------------------
 
\documentclass[18pt]{article}
 
\usepackage[margin=1in]{geometry} 
\usepackage{amsmath,amsthm,amssymb}
\usepackage[margin=1in]{geometry} 
\usepackage{amsmath,amsthm,amssymb}
\usepackage[T1]{fontenc} %escribe lo del teclado
\usepackage[utf8]{inputenc} %Reconoce algunos símbolos
\usepackage{lmodern} %optimiza algunas fuentes
\usepackage{graphicx}
\graphicspath{ {images/} }
\usepackage{hyperref} % Uso de links
\usepackage[colorinlistoftodos]{todonotes}
 
\newcommand{\N}{\mathbb{N}}
\newcommand{\Z}{\mathbb{Z}}
 
\newenvironment{theorem}[2][Theorem]{\begin{trivlist}
\item[\hskip \labelsep {\bfseries #1}\hskip \labelsep {\bfseries #2.}]}{\end{trivlist}}
\newenvironment{lemma}[2][Lemma]{\begin{trivlist}
\item[\hskip \labelsep {\bfseries #1}\hskip \labelsep {\bfseries #2.}]}{\end{trivlist}}
\newenvironment{exercise}[2][Exercise]{\begin{trivlist}
\item[\hskip \labelsep {\bfseries #1}\hskip \labelsep {\bfseries #2.}]}{\end{trivlist}}
\newenvironment{problem}[2][Problem]{\begin{trivlist}
\item[\hskip \labelsep {\bfseries #1}\hskip \labelsep {\bfseries #2.}]}{\end{trivlist}}
\newenvironment{question}[2][Question]{\begin{trivlist}
\item[\hskip \labelsep {\bfseries #1}\hskip \labelsep {\bfseries #2.}]}{\end{trivlist}}
\newenvironment{corollary}[2][Corollary]{\begin{trivlist}
\item[\hskip \labelsep {\bfseries #1}\hskip \labelsep {\bfseries #2.}]}{\end{trivlist}}

\newenvironment{solution}{\begin{proof}[Solution]}{\end{proof}}
 
\begin{document}
 
% --------------------------------------------------------------
%                         Start here
% --------------------------------------------------------------
 
\title{READING NOTES: What is Theory?}
\author{Zhilong Wang}

\maketitle
\section{Summary}
\subsection{Technology and Social Theory}
Chapter 1 talk about "what technology is, what technology does and how technology has been theorized
and what we as social theoriests should be mindful of when studying it."\cite{textbook}
To define technology, the author demonstrates from three views: objects, activities and knowledge.
% As our discussion progresses we add new layers
% of complexity. We move from individual tools and objects to machines,
% buildings, sociotechnical systems and companion species, and to thinking
% of such technologies as ways of ordering worlds rather than simply as ob-
% jects in that world. Additionally, then, we come to think about technologies
% as modes of organization.

Technology is slippery(difficulty of defining technology):
why:
1.Keeping transition and multiple uses and meanings.
2.What we understand by the world has changed across time.

\subsection{What technology is?}
Technology was seen as:
\begin{enumerate}
    \item Physical things:objects, artifacts, tools, machines and so on;
    \item Human activities;
    \item Knowledge.
    \item A mode of social organization.
    \item Sociotechnical systems.
\end{enumerate}
"For example, you are currently reading this
chapter. To do so requires an object (this book), an activity (reading) and
knowledge (of the English language)."\cite{textbook}

\subsection{What technology does?}
\begin{enumerate}
    \item help us adapt to or control environments
    \item solve problems (and create new ones)
    \item extend human forces and senses
    \item mediate between the physical world and the cultural one
    \item are modes of being and knowing, revealing and enframing
    \item are agents.
\end{enumerate}

\subsection{What has technology been theorized?}
\begin{enumerate}
    \item by privileging technology
    \item by privileging society
    \item by thinking about the mutual entanglement of technology, society and other things besides.
\end{enumerate}

\textbf{path dependency}Our decisions about technologies was influence by the decisions in history.
\subsection{Technology, Systems and Social Interests}
\begin{enumerate}
    \item think beyond the lone genius inventor
    \item include considerations of power, capital and the ability to persuade – they are all important factors in technological success
    \item look to the positively reinforcing interactions that sustain sociotechnical systems
    \item appreciate previous events (that past informs the present) and their potentially reinforcing nature (positive or negative feedback).
\end{enumerate}

\subsection{Our Times: Technology, Complexity and Risk}
The development of technology make it harder and harder for people to comprehend the very technologies
that constitute our environment.
\begin{enumerate}
    \item modern technologies are extensive and open-textured, even experts may struggle to master them
    \item the intended outcomes for technology might not work out in realityTheorizing Technology
    \item technologies are real-time experiments, they have revenge effects, they are accidents waiting to happen
    \item ours is a world of technologically-induced global risk.
\end{enumerate}

\subsection{What theory is not}
\textbf{Not theory:} references, data, variables, diagrams, and hypotheses.

A good theory explains, predicts, and delights.

\section{Critique}



% --------------------------------------------------------------
%     You don't have to mess with anything below this line.
% --------------------------------------------------------------
\bibliographystyle{IEEEtranS}
\bibliography{ref}
\end{document}
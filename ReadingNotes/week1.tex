% --------------------------------------------------------------
% This is all preamble stuff that you don't have to worry about.
% Head down to where it says "Start here"
% --------------------------------------------------------------
 
\documentclass[18pt]{article}
 
\usepackage[margin=1in]{geometry} 
\usepackage{amsmath,amsthm,amssymb}
\usepackage[margin=1in]{geometry} 
\usepackage{amsmath,amsthm,amssymb}
\usepackage[T1]{fontenc} %escribe lo del teclado
\usepackage[utf8]{inputenc} %Reconoce algunos símbolos
\usepackage{lmodern} %optimiza algunas fuentes
\usepackage{graphicx}
\graphicspath{ {images/} }
\usepackage{hyperref} % Uso de links
\usepackage[colorinlistoftodos]{todonotes}
 
\newcommand{\N}{\mathbb{N}}
\newcommand{\Z}{\mathbb{Z}}
 
\newenvironment{theorem}[2][Theorem]{\begin{trivlist}
\item[\hskip \labelsep {\bfseries #1}\hskip \labelsep {\bfseries #2.}]}{\end{trivlist}}
\newenvironment{lemma}[2][Lemma]{\begin{trivlist}
\item[\hskip \labelsep {\bfseries #1}\hskip \labelsep {\bfseries #2.}]}{\end{trivlist}}
\newenvironment{exercise}[2][Exercise]{\begin{trivlist}
\item[\hskip \labelsep {\bfseries #1}\hskip \labelsep {\bfseries #2.}]}{\end{trivlist}}
\newenvironment{problem}[2][Problem]{\begin{trivlist}
\item[\hskip \labelsep {\bfseries #1}\hskip \labelsep {\bfseries #2.}]}{\end{trivlist}}
\newenvironment{question}[2][Question]{\begin{trivlist}
\item[\hskip \labelsep {\bfseries #1}\hskip \labelsep {\bfseries #2.}]}{\end{trivlist}}
\newenvironment{corollary}[2][Corollary]{\begin{trivlist}
\item[\hskip \labelsep {\bfseries #1}\hskip \labelsep {\bfseries #2.}]}{\end{trivlist}}

\newenvironment{solution}{\begin{proof}[Solution]}{\end{proof}}
 
\begin{document}
 
% --------------------------------------------------------------
%                         Start here
% --------------------------------------------------------------
 
\title{READING NOTES: What is Science?}
\author{Zhilong Wang}

\maketitle
\section{Summary}
\subsection{The Prehistory of Science and Technology Studies}
\subsubsection{What is science?}
What is science is an important topic for the scientists. 
Thare are many different views of science regarding this problem.
From the view of philosophies, there are two important understanding of scientific worldview
--\textbf{logical positivism} and \textbf{falsificationism}. 

Logical positivists hold the view that scientific theories come from the obervations of the natural world\cite{ayer2012language,richardson1998carnap}.
Scientific theories are general claims developed by scientists to represent the data comes from natural world.

Falsificationism believes that good scientific theories are one that can be used to predict the natural world, that is to mean, falsifiable.
The creation of scientific is more like a "imaginative creations" than a "observations from the world".
In the case, scientific theories are only temperarily accept until contradictive observations are found.

Besides, there are many other important features for the scientific theories--\textbf{progress} and \textbf{truths}.
\textbf{progress} and \textbf{truths} mean that science should be able to advance and approach the truths along with its development.

\subsubsection{What is technology?}
While science tell the theories to predict that how thing will work, 
technology apply the scientific theories to a certain situation.
Technology usually is the combination of the scientific theories and creative ideas\cite{sismondo2010introduction}.
But, technology is limited by the scientific knowledge because the science are the basics of the technology. 
Regarding the effects of the techonloty, people are more care about their contribution to the socienty and humans.

There are two kinds of technology: "life-oriented" technology sevice for the human needs 
and "mega machines" technology provide more power by regimenting and dehumanizing\cite{sismondo2010introduction}. 
The advance of technology is a double-edged sword, which may not only contribute to human welfare and well-being,
but also lead to a threaten. For instance, when Einstein's theory of relativity was used to direct
the inventaton of the nuclear weapons, it becomes a threaten. In which way the science and the technologies 
are used usually lead to very different effects. 
The project of "Science, Technology and Society" was launched to promiot the social responsibility of scientists and researchers.

\subsection{The Kuhnian Revolution}
Thomas kuhn's book, The Structure of Scientific Revolutions poses a challenge to 
traditional philosophical view of the history of science.
Normal science was viewed as a steady process of "development-by-accumulation"\cite{hellter} of accepted facts and theories.
Kuhn rejected traditional view of steady progress, that scientific research attempts to step forward from previous view,
and argued that science is only what the scientists do, following their causes.

In normal science, scientific research shares a recognition of fundemental principles, beliefs and understanding in their fields. 
In Kuhn's view, scientific research share a gramatical pattern, and scientific achievement
can be shared between different researchers.
Normal science researchers view the development of science as a well organized process. 
In the theorietical side of paradigm, they provide frameworks to clarify the phenomena;
In the practical side of a paradigm, they provide a "form of life" to instruct the behavior and action.
Different from the normal science, which continuously solves the difficult problems,
Kuhn's view are more skill at socializing parctice. 
In the normal science, different researchers come from different areas can communicate and recognizing 
each other's research questions and achievements freely. However, Kuhn believes that science theories in different
areas are incommensurable for three reasons: 
Firstly, meaning of theoretical terms are related to abservations they imply;
Secondly, paradigms can shape observations;
Thirdly, past research are so opaque and bizarre that it is difficult to make use of past scientific problems,
concepts and methods.

In 1997, Peter Galison proposed that new detectors can be used in the same kind of research and observations.
No immediate change will happen after the cheories change. Discontinuity is generally connected to others.

In the communicating among social worlds, in most case, it is believed that the disciplines from different units
are incommensurable. There are two concepts to understanding communication between different units:
"trading zone" and "boundary objects". A "trading zone" exists, in which researcher can interact via simplified languages or 
pidgins without full assimilation; A "boundary objects" can be formed as a bridges acrosss different groups
and help to focus the attention between them.
\section{Critique}
The problem of prositivism is that scientists can draw different theories from the same data;
The problem of the falsificationism is that 
as scientific theories are usually abstract representations of the natural rules, 
they can hardly be overturned. Excuses can always be found to coordinate the conflict between the
abstract theories and reality.

Most philosophers, liguists disagree with the case for semantic incommensurability proposed by Kuhn.
They think that semantic incommensurability cannot be sustained, or impossible.
After the claims of radical incommensurability was proved to be fail, Kuhn explained that the "incommensurability" 
only represents "incomplete communication" or "difficulty of translation".

\section{My Experience with Science and Technology}
By reading this material, I am more clear about what is science and what is technology.
It get me to think about aims of my research. 
My research interests focus on system and software security, which can be group in to the categary of technologies--attacking and defence. 
On the one hand, if my research works in defence technolgy were applied the industry, it will make secure
people's important information and proprities, and thus, enhance their human welfare and well-being.
On the other hand, if I apply my research to the attack fields, it can be a threaten for countries and society.
So, it is so important that we must think about the effects of our research because it may result to a unanticipated consquence.
I also believe that "trading zone" exists between different groups or units that can help them to understand each other even it is hard.
Researchers can benefit from the communication between different areas.
For instance, the combination of information technology and other fields has already make
our society become better and better. Examples include but not limited to the Internet of Things, Computer Aided Designed and so on.   
% --------------------------------------------------------------
%     You don't have to mess with anything below this line.
% --------------------------------------------------------------
\bibliographystyle{IEEEtranS}
\bibliography{ref}
\end{document}